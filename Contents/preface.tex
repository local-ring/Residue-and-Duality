%%%%%%%%%%%%%%%%%%%%%%preface.tex%%%%%%%%%%%%%%%%%%%%%%%%%%%%%%%%%%%%%%%%%
% sample preface
%
% Use this file as a template for your own input.
%
%%%%%%%%%%%%%%%%%%%%%%%% Springer %%%%%%%%%%%%%%%%%%%%%%%%%%

\preface

%% Please write your preface here
In the spring of 1963 I suggested to Grothendieck the possibility of my running a seminar at Harvard on his theory of duality for coherent sheaves -- a theory which had been hinted at in his talk to S\'eminaire Bourbaki in 1957 \cite{g1957}, and in his talk to the International Congress of Mathematicians in 1958 \cite{g1958}, but had never been developed systematically. He agreed, saying that he would provide an outline of the material, if I would fill in the details and write up lecture notes of the seminar. During the summer of 1963, he wrote a series of ``pr\'enotes'' which were to be the basis for the seminar.\par
I quote from the preface of the pr\'enotes:
\begin{quote}
    Les presentes notes donnent une esquisse assez d\'{e}taill\'{e}e d'une th\'{e}orie cohomologique de la dualit\'{e} des Modules coh\'{e}rents sur les pr\'{e} sch\'{e}mas. Les id\'{e}es principales de la th\'{e}orie m'etaient connues des 1959, mais le manque de fondements adequats d'Alg\`{e}bre Homologique m'avait emp\^{e}ch\'{e} d'aborder une redaction d'ensemble. Cette lacune de fondements est sur le point d'etre combl\'{e}e par la th\`{e}se de VERDIER, ce qui rend en principe possible un expos\'{e} satisfaisant. Il est d'ailleurs apparu depuis qu il existe des th\'{e}ories cohomologiques de dualit\'{e} formellement tr\`{e}s analogues a celle d\'{e}velopp\'{e}e ici dans toutes sortes d'autres contextes: faisceaux coherents sur les espaces analytiques, faisceaux ab\'{e}liens sur les espaces topologiques (VERDIER), modules galoisiens (VERDIER, TATE), faisceaux de torsion sur les sch\'{e}mas munis de leur topologie \'{e}tale, corps de classe en tous genres... Cela me semble une raison assez s\'{e}rieuse pour se familiariser avec le yoga g\'{e}n\'{e}ral de la dualit\'{e} dans un cas type, comme la th\'{e}orie cohomologique des residus.\par
    La th\'{e}orie consiste pour l'essentiel dans des questions de variance: construction d'un foncteur $f^{!}$ et d'un homomorphisme-trace \[Rf_{\ast}f^{!}\rightarrow id.\] La construction donn\'{e}e ici est compliqu\'{e}e et indirecte et n'est pas valable sous des conditions aussi g\'{e}n\'{e}rales qu'on est en droit de s'y attendre. Il faudra sans doute une id\'{e}e nouvelle pour apporter des simplifications substantielles.
\end{quote}

The seminar took place in the fall and winter of 1963-64, with the assistance of David Mumford, John Tate, Stephen Lichtenbaum, John Fogarty, and others, and gave rise to a series of six expos\'es which were circulated to a limited audience under the title ``S\'eminaire Hartshorne''. The present notes are a revised, expanded, and completed version of the prevoius notes.\par
I would like to take this opportunity to thank all those people who have helped in the course of this work, and in particular A. Grothendieck, who gave continual support and encouragement throughout the whole project.




\vspace{\baselineskip}
\begin{flushright}\noindent
\hfill {\it R. H.}\\
Cambridge, May 1966
\end{flushright}




\Extrachap{Introduction}
% \epigraph{I recall seeing a package to make quotes}{Snowball}
The main purpose of these is to prove a duality theorem for the cohomology of quasi-coherent sheaves, with respect to a proper morphism of locally noetherian preschemes. Various such theorems are already known. Typical is the duality theorem for a non-singular complete curve $X$ over an algebraically closed field $k$, which say that
\[h^0(D)=h^1(K-D),\]
where $D$ is a divisor, $K$ is the canonical divisor, and 
\[h^{i}(D)=\dim_k H^i(X,L(D))\]
for any $i$, and any divisor $D$. (See e.g. \cite{s1959} Ch. II for a proof.) \par

Various attempts were made to generalize this theorem to varieties of higher dimension, and as Zariski points out in his report \cite{report1956}, his generalization of a lemma of Enriques-Severi \cite{z1952} is equivalent to the statement that for a normal projective variety $X$ of dimension $n$ over $k$,
\[h^0(D)=h^n(K-D)\]
for any divisor $D$. This is also equivalent to a theorem of Serre (See \cite{fac} 76 Thm. 4) on the vanishing of the cohomology group $H^1(X,L(-m))$ for $m$ large and $L$ locally free. Using a related theorem (See \cite{fac} 75 Thm. 3), Zariski shows how one can deduce on a non-singular projective variety the formula
\[h^i(D)=h^{n-i}(K-D)\]
for $0\leq i\leq n$. In terms of sheaves, this result corresponds to the fact that the $k$-vector spaces
\[H^i(X,F)\quad  \texttt{ and } \quad H^{n-i}(X,F^{\vee}\otimes \omega)\]
are dual to each other, where $F$ is locally free sheaf, $F^{\vee}$ is the dual sheaf $\uHom(F,\OO_X)$, and $\omega=\Omega^n_{X/k}$ is the sheaf of $n$-differentials on $X$.
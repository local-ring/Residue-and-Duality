\Extrachap{Introduction}
% \epigraph{I recall seeing a package to make quotes}{Snowball}
The main purpose of these is to prove a duality theorem for the cohomology of quasi-coherent sheaves, with respect to a proper morphism of locally noetherian preschemes. Various such theorems are already known. Typical is the duality theorem for a non-singular complete curve $X$ over an algebraically closed field $k$, which say that
\[h^0(D)=h^1(K-D),\]
where $D$ is a divisor, $K$ is the canonical divisor, and 
\[h^{i}(D)=\dim_k H^i(X,L(D))\]
for any $i$, and any divisor $D$. (See e.g. \cite{s1959} Ch. II for a proof.) \par

Various attempts were made to generalize this theorem to varieties of higher dimension, and as Zariski points out in his report \cite{report1956}, his generalization of a lemma of Enriques-Severi \cite{z1952} is equivalent to the statement that for a normal projective variety $X$ of dimension $n$ over $k$,
\[h^0(D)=h^n(K-D)\]
for any divisor $D$. This is also equivalent to a theorem of Serre (See \cite{fac} 76 Thm. 4) on the vanishing of the cohomology group $H^1(X,L(-m))$ for $m$ large and $L$ locally free. Using a related theorem (See \cite{fac} 75 Thm. 3), Zariski shows how one can deduce on a non-singular projective variety the formula
\[h^i(D)=h^{n-i}(K-D)\]
for $0\leq i\leq n$. In terms of sheaves, this result corresponds to the fact that the $k$-vector spaces
\[H^i(X,\F)\quad \textrm{ and } \quad H^{n-i}(X,\F^{\vee}\otimes \omega)\]
are dual to each other, where $\F$ is locally free sheaf, $F^{\vee}$ is the dual sheaf $\uHom(\F,\OO_X)$, and $\omega=\Omega^n_{X/k}$ is the sheaf of $n$-differentials on $X$. Serre gives a proof of this same theorem by analytic methods for a compact complex analytic manifold $X$.\par

Grothendieck gave some generalizations of these theorems for non-singular projective varieties, and then in \cite{} announced the general theorem for schemes proper over a field, with arbitrary singularities, which is the subject of the present lecture notes.\par 

To motivate the statement of our main theorem, let us consider the case of projective space $X=\PP^n_k$ over an algebraically closed field $k$. Then there is a canonical isomorphism
\[H^n(X,\omega)\cong k\]
where $\omega^n_{X/k}$ is the sheaf of $n$-differentials. Combining this with the Yoneda pairing
\[H^i(X,\F)\times \Ext_X^{n-i}(\F,\omega)\rightarrow H^n(X,\omega)\]
we obtain a pairing
\[H^i(X,\F)\times \Ext^{n-i}_X(\F,\omega)\rightarrow k\]
which one shows easily to be a perfect pairing \cite{}. This genarlizes the statements above, becasue for a locally free sheaf $\F$,
\[\Ext^{n-i}(\F,\omega)=\Ext^{n-i}(\OO_X,\F^{\vee}\otimes \omega)=H^{n-i}(X,\F^{\vee}\otimes \omega).\]
Another way of looking at our duality pairing is as an isomorphism
\[\Ext^{n-i}_X (\F,\omega)\rightarrow \Hom_k(H^i(X,\F),k).\]
Since everything is linear over $k$, we may introduce a $k$-vector space $G$, and have an isomorphism
\[\Ext^{n-i}_X (\F,G\otimes_k\omega)\rightarrow \Hom_k(H^i(X,\F),G).\]
Before proceeding further, we must introduce the derived category. It will be discussed in detail in Chapter I, but for the moment it will be sufficient to know the following:

For each abelian category $\A$, there is a category $D(\A)$, called the {\it derived category} of $A$, whose objects are complexes of objects of $A$. If $F:\A\rightarrow \B$ is an additive functor from one abelian category to another, then under reasonable conditions there is a {\it right derived functor} \[RF:D(\A)\rightarrow D(\B)\] 
with the property that for any $X\in Ob(\A)$, if $X$ denotes also the complex which is $X$ in degree zero, and zero elsewhere, then $H^i(RF(X))=R^iF(X)$, where $R^iF$ is the ordinary $i$-th right derived functor of $F$. Finally, if $F:\A\to \B$ and $G:\B\to \CC$ are two functors then 
\[R(G\circ F)=R(G)\circ R(F).\]
This replaces the old-fashioned spectral sequence of a composite functor.

Now we can jazz up our duality for projective space as follows. We replace $k$ by a prescheme $Y$, so that $X=\PP^n_Y$. We consider the derived categories $D(X)$ and $D(Y)$ of the categories of $\OO_X$-modules and $\OO_Y$-modules, respectively. Then cohomology $H^i$ becomes $Rf_*$, the derived functor of the direct image functor $f_*$, where $f:X\to Y$ is the projection. $\Ext$

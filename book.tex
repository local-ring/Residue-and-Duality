%%%%%%%%%%%%%%%%%%%% book.tex %%%%%%%%%%%%%%%%%%%%%%%%%%%%%
%
% sample root file for the chapters of your "monograph"
%
% Use this file as a template for your own input.
%
%%%%%%%%%%%%%%%% Springer-Verlag %%%%%%%%%%%%%%%%%%%%%%%%%%


% RECOMMENDED %%%%%%%%%%%%%%%%%%%%%%%%%%%%%%%%%%%%%%%%%%%%%%%%%%%
\documentclass[graybox,envcountchap,sectrefs]{svmono}

% choose options for [] as required from the list
% in the Reference Guide

\usepackage{mathptmx}
\usepackage{amsmath}
\usepackage{helvet}
\usepackage{courier}
%
\usepackage{type1cm} 
\usepackage[mathcal]{euscript}    

\usepackage{makeidx}         % allows index generation
\usepackage{graphicx}        % standard LaTeX graphics tool
                             % when including figure files
\graphicspath{{Figures/}}
\usepackage{multicol}        % used for the two-column index
\usepackage[bottom]{footmisc}% places footnotes at page bottom

% see the list of further useful packages
% in the Reference Guide

\makeindex             % used for the subject index
                       % please use the style svind.ist with
                       % your makeindex program
    \addtolength{\oddsidemargin}{-1.25in}
 	\addtolength{\evensidemargin}{-1in}
	\addtolength{\textwidth}{2.5in}

	\addtolength{\topmargin}{0in}
	\addtolength{\textheight}{1.5in}
	
\usepackage{epigraph}
%%%%%%%%%%%%%%%%%%%%%%%%%%%%%%%%%%%%%%%%%%%%%%%%%%%%%%%%%%%%%%%%%%%%%

\begin{document}

\author{Robin Hartshorne}
\title{Residues and Duality}
\subtitle{based on a seminar on the work of A. Grothendieck}
\maketitle

\frontmatter%%%%%%%%%%%%%%%%%%%%%%%%%%%%%%%%%%%%%%%%%%%%%%%%%%%%%%
\newcommand{\N}{\mathbb{N}}
\newcommand{\R}{\mathbb{R}}
\newcommand{\Q}{\mathbb{Q}}
\newcommand{\Z}{\mathbb{Z}}

\newcommand{\C}{\mathbf{C}}
\newcommand{\PP}{\mathbf{P}}
\newcommand{\HHH}{\mathbf{H}}
\newcommand{\DD}{\mathbf{D}}

\newcommand{\OO}{\mathcal{O}}
\newcommand{\B}{\mathcal{B}}
\newcommand{\cU}{\mathcal{U}}
\newcommand{\F}{\mathcal{F}}
\newcommand{\G}{\mathcal{G}}
\newcommand{\A}{\mathcal{A}}
\newcommand{\CC}{\mathcal{C}}
\newcommand{\DDD}{\mathcal{D}}



\newcommand{\rO}{\mathrm{O}}
\newcommand{\op}{\mathrm{op}}
\newcommand{\rd}{~\mathrm{d}} %roman d
\newcommand{\GL}{\mathrm{GL}}
\newcommand{\SU}{\mathrm{SU}}
\newcommand{\SL}{\mathrm{SL}}
\newcommand{\SO}{\mathrm{SO}}
\newcommand{\U}{\mathrm{U}}
\newcommand{\Sp}{\mathrm{Sp}}
\newcommand{\im}{\mathrm{Im}}
\newcommand{\ord}{\mathrm{ord}}
\newcommand{\End}{\mathrm{End}}
\newcommand{\Aut}{\mathrm{Aut}}
\newcommand{\Hom}{\mathrm{Hom}}
\newcommand{\ad}{\mathrm{ad}}
\newcommand{\tr}{\mathrm{tr}}
\newcommand{\Rad}{\mathrm{Rad}}
\newcommand{\Ext}{\mathrm{Ext}}
\newcommand{\Tor}{\mathrm{Tor}}

\newcommand{\uHom}{\underline{\mathrm{Hom}}}

% \newcommand{\DD}{\overline{\mathbf{D}}}


\newcommand{\res}{\mathsf{res}}



% \include{Contents/dedic}
% \include{Contents/foreword}
%%%%%%%%%%%%%%%%%%%%%%preface.tex%%%%%%%%%%%%%%%%%%%%%%%%%%%%%%%%%%%%%%%%%
% sample preface
%
% Use this file as a template for your own input.
%
%%%%%%%%%%%%%%%%%%%%%%%% Springer %%%%%%%%%%%%%%%%%%%%%%%%%%

\preface

%% Please write your preface here
In the spring of 1963 I suggested to Grothendieck the possibility of my running a seminar at Harvard on his theory of duality for coherent sheaves -- a theory which had been hinted at in his talk to S\'eminaire Bourbaki in 1957 \cite{g1957}, and in his talk to the International Congress of Mathematicians in 1958 \cite{g1958}, but had never been developed systematically. He agreed, saying that he would provide an outline of the material, if I would fill in the details and write up lecture notes of the seminar. During the summer of 1963, he wrote a series of ``pr\'enotes'' which were to be the basis for the seminar.\par
I quote from the preface of the pr\'enotes:
\begin{quote}
    Les presentes notes donnent une esquisse assez d\'{e}taill\'{e}e d'une th\'{e}orie cohomologique de la dualit\'{e} des Modules coh\'{e}rents sur les pr\'{e} sch\'{e}mas. Les id\'{e}es principales de la th\'{e}orie m'etaient connues des 1959, mais le manque de fondements adequats d'Alg\`{e}bre Homologique m'avait emp\^{e}ch\'{e} d'aborder une redaction d'ensemble. Cette lacune de fondements est sur le point d'etre combl\'{e}e par la th\`{e}se de VERDIER, ce qui rend en principe possible un expos\'{e} satisfaisant. Il est d'ailleurs apparu depuis qu il existe des th\'{e}ories cohomologiques de dualit\'{e} formellement tr\`{e}s analogues a celle d\'{e}velopp\'{e}e ici dans toutes sortes d'autres contextes: faisceaux coherents sur les espaces analytiques, faisceaux ab\'{e}liens sur les espaces topologiques (VERDIER), modules galoisiens (VERDIER, TATE), faisceaux de torsion sur les sch\'{e}mas munis de leur topologie \'{e}tale, corps de classe en tous genres... Cela me semble une raison assez s\'{e}rieuse pour se familiariser avec le yoga g\'{e}n\'{e}ral de la dualit\'{e} dans un cas type, comme la th\'{e}orie cohomologique des residus.\par
    La th\'{e}orie consiste pour l'essentiel dans des questions de variance: construction d'un foncteur $f^{!}$ et d'un homomorphisme-trace \[Rf_{\ast}f^{!}\rightarrow id.\] La construction donn\'{e}e ici est compliqu\'{e}e et indirecte et n'est pas valable sous des conditions aussi g\'{e}n\'{e}rales qu'on est en droit de s'y attendre. Il faudra sans doute une id\'{e}e nouvelle pour apporter des simplifications substantielles.
\end{quote}

The seminar took place in the fall and winter of 1963-64, with the assistance of David Mumford, John Tate, Stephen Lichtenbaum, John Fogarty, and others, and gave rise to a series of six expos\'es which were circulated to a limited audience under the title ``S\'eminaire Hartshorne''. The present notes are a revised, expanded, and completed version of the prevoius notes.\par
I would like to take this opportunity to thank all those people who have helped in the course of this work, and in particular A. Grothendieck, who gave continual support and encouragement throughout the whole project.




\vspace{\baselineskip}
\begin{flushright}\noindent
\hfill {\it R. H.}\\
Cambridge, May 1966
\end{flushright}




% \include{Contents/acknow}


\tableofcontents

% \include{Contents/acronym}


\mainmatter%%%%%%%%%%%%%%%%%%%%%%%%%%%%%%%%%%%%%%%%%%%%%%%%%%%%%%%
% \include{Contents/part}
\Extrachap{Introduction}
% \epigraph{I recall seeing a package to make quotes}{Snowball}
The main purpose of these is to prove a duality theorem for the cohomology of quasi-coherent sheaves, with respect to a proper morphism of locally noetherian preschemes. Various such theorems are already known. Typical is the duality theorem for a non-singular complete curve $X$ over an algebraically closed field $k$, which say that
\[h^0(D)=h^1(K-D),\]
where $D$ is a divisor, $K$ is the canonical divisor, and 
\[h^{i}(D)=\dim_k H^i(X,L(D))\]
for any $i$, and any divisor $D$. (See e.g. \cite{s1959} Ch. II for a proof.) \par

Various attempts were made to generalize this theorem to varieties of higher dimension, and as Zariski points out in his report \cite{report1956}, his generalization of a lemma of Enriques-Severi \cite{z1952} is equivalent to the statement that for a normal projective variety $X$ of dimension $n$ over $k$,
\[h^0(D)=h^n(K-D)\]
for any divisor $D$. This is also equivalent to a theorem of Serre (See \cite{fac} 76 Thm. 4) on the vanishing of the cohomology group $H^1(X,L(-m))$ for $m$ large and $L$ locally free. Using a related theorem (See \cite{fac} 75 Thm. 3), Zariski shows how one can deduce on a non-singular projective variety the formula
\[h^i(D)=h^{n-i}(K-D)\]
for $0\leq i\leq n$. In terms of sheaves, this result corresponds to the fact that the $k$-vector spaces
\[H^i(X,F)\quad  \texttt{ and } \quad H^{n-i}(X,F^{\vee}\otimes \omega)\]
are dual to each other, where $F$ is locally free sheaf, $F^{\vee}$ is the dual sheaf $\uHom(F,\OO_X)$, and $\omega=\Omega^n_{X/k}$ is the sheaf of $n$-differentials on $X$.
\include{Contents/ch1}
% \include{Contents/ch2}
% \include{Contents/appendix}

\backmatter%%%%%%%%%%%%%%%%%%%%%%%%%%%%%%%%%%%%%%%%%%%%%%%%%%%%%%%

% \include{Contents/solutions}
% \printindex

%%%%%%%%%%%%%%%%%%%%%%%%%%%%%%%%%%%%%%%%%%%%%%%%%%%%%%%%%%%%%%%%%%%%%%
\biblstarthook{}

\begin{thebibliography}{99.}%
% and use \bibitem to create references.
%
% Use the following syntax and markup for your references if 
% the subject of your book is from the field 
% "Mathematics, Physics, Statistics, Computer Science"
%

\bibitem{g1957} Grothendieck, Alexander. "Théorèmes de dualité pour les faisceaux algébriques cohérents." Séminaire Bourbaki 149 (1957): 25.
\bibitem{g1958} Grothendieck, Alexander. "The cohomology theory of abstract algebraic varieties." Proceedings of the International Congress of Mathematicians. 1958.
\bibitem{s1959} Serre, J-P. "Groupes algébriques et corps de classes." Hermann (1959).
\bibitem{report1956} Zariski, Oscar. "Scientific report on the second summer institute, several complex variables. Part III. Algebraic sheaf theory." Bulletin of the American Mathematical Society 62.2 (1956): 117-141.
\bibitem{z1952} Zariski, Oscar. "Complete linear systems on normal varieties and a generalization of a lemma of Enriques-Severi." Annals of Mathematics (1952): 552-592.
\bibitem{fac} Serre, Jean-Pierre. "Faisceaux algébriques cohérents." Annals of Mathematics (1955): 197-278.
\end{thebibliography}

\end{document}





